% Created 2024-10-24 Thu 13:09
% Intended LaTeX compiler: xelatex
\documentclass[a4paper,oneside,toc=bibliography,toc=listof]{scrbook}
\usepackage{graphicx}
\usepackage{longtable}
\usepackage{wrapfig}
\usepackage{rotating}
\usepackage[normalem]{ulem}
\usepackage{capt-of}
\usepackage{hyperref}
\usepackage[T1]{fontenc}
\usepackage[utf8]{inputenc}
\usepackage{amsmath, amssymb}
\usepackage{bm}
\usepackage{booktabs}
\usepackage{longtable}
\usepackage[printonlyused, smaller]{acronym}
\usepackage{subcaption}
\usepackage{placeins}
\usepackage{tikz}
\usepackage{pgfplots}
\pgfplotsset{ compat = newest, grid=major, every axis plot/.append style={very thick}, }
\usetikzlibrary{calc,fit, positioning,arrows.meta}
\tikzset{>={Latex[width=2mm,length=2mm]}} % more visible default arrow heads
\tikzstyle{block} = [draw=black, fill=white, rectangle, align=center, minimum height=2em, minimum width=3em]
\tikzstyle{sum} = [draw, circle, node distance=1cm]
\newlength\figureheight
\newlength\figurewidth
\setlength\figureheight{3cm}
\setlength\figurewidth{0.7\textwidth}
\RequirePackage{xcolor}
\definecolor{UStuttDarkBlue}{RGB}{0,81,158}
\definecolor{UStuttLightBlue}{RGB}{0,190,255}
\definecolor{UStuttDarkGreen}{RGB}{59,140,122}
\definecolor{UStuttLightGreen}{RGB}{125,155,101}
\definecolor{UStuttDarkOrange}{RGB}{228,175,52}
\definecolor{UStuttLightOrange}{RGB}{236,218,145}
\usepackage{listings}
\usepackage{scrhack} % if you load listings together with scrbook etc., then load this fixing package as well
\lstset{
basicstyle=\ttfamily\small,  % Set smaller font size
captionpos=b,
commentstyle=\color{UStuttDarkGreen},
frame=single,                    % adds a frame around the code
keepspaces=true,
%keywordstyle=\color{UStuttDarkBlue},
showspaces=false,
showstringspaces=false,          % underline spaces within strings only
showtabs=false,
stringstyle=\color{UStuttDarkBlue},
tabsize=2
}
\PassOptionsToPackage{pdfpagelabels}{hyperref}
\usepackage{hyperref}  % backref linktocpage pagebackref
\ifpdf
\pdfcompresslevel=9
\pdfadjustspacing=1
\fi
\hypersetup{%
%draft, % = no hyperlinking at all
%colorlinks=true,
colorlinks=false,
linktocpage=false, pdfborder={0 0 0},%
breaklinks=true, pdfpagemode=UseNone, pageanchor=true, pdfpagemode=UseOutlines,%
plainpages=false, bookmarksnumbered, bookmarksopen=true, bookmarksopenlevel=1,%
hypertexnames=true, pdfhighlight=/O,%nesting=true,%frenchlinks,%
%urlcolor=Black, linkcolor=Black, citecolor=Black, %pagecolor=Black,%
}
\newcommand{\R}{\mathbb{R}}
\newcommand{\T}{\mathrm{T}}
\newcommand{\mustbe}{\ensuremath{\stackrel{!}{=}}}
\newcommand{\bmat}[1]{ \ensuremath{\begin{bmatrix} #1 \end{bmatrix}} }
\newcommand{\partfrac}[2]{ \ensuremath{\frac{\partial #1}{\partial #2}} }
\newcommand{\ddiff}{\ensuremath{\mathrm{d}}}
\newcommand{\ddt}{\ensuremath{\frac{\ddiff}{\ddiff t}}}
\usepackage{lipsum}
\usepackage{fontspec}
\lstdefinelanguage{org}{keywords={BEGIN,END,CAPTION,NAME,src,results},morecomment=[l]{\#},morestring=[b]"}
\providecommand{\lstlistoflistings}{}
\usepackage{changepage}  % To adjust width
\newenvironment{narrowmargin}{
\begin{center}
\begin{adjustwidth}{-0.5cm}{-0.5cm}
}{
\end{adjustwidth}
\end{center}
}
\newenvironment{mediummargin}{
\begin{center}
\begin{adjustwidth}{-1cm}{-1cm}
}{
\end{adjustwidth}
\end{center}
}
\newenvironment{widemargin}{
\begin{center}
\begin{adjustwidth}{-1.5cm}{-1.5cm}
}{
\end{adjustwidth}
\end{center}
}
\newenvironment{verywidemargin}{
\begin{center}
\begin{adjustwidth}{-2cm}{-2cm}
}{
\end{adjustwidth}
\end{center}
}
\newenvironment{extrawidemargin}{
\begin{center}
\begin{adjustwidth}{-2.5cm}{-2.5cm}
}{
\end{adjustwidth}
\end{center}
}
\usepackage[
type=MA
]{iswthesis}
\usepackage[ngerman, english]{babel}
\addbibresource{bibliography.bib}
\title{Emacs Orgmode ISW Template}
\titleTranslated{Emacs Orgmode ISW Template}
\author{Max Mustermann}
\placeOfBirth{Stuttgart}
\address{Seidenstraße 36, 70174 Stuttgart}
\major{Mechatronik}
\matrnr{1234567}
\date{\today}
\supervisor{My supervisor, M.Sc.}
\professor{Prof. Dr.-Ing. Oliver Riedel}
\date{\today}
\title{Emacs Orgmode ISW Template}
\hypersetup{
 pdfauthor={},
 pdftitle={Emacs Orgmode ISW Template},
 pdfkeywords={},
 pdfsubject={},
 pdfcreator={Emacs 29.4 (Org mode 9.7.11)}, 
 pdflang={English}}
\begin{document}

\frontmatter
\makeISWtitle
\cleardoublepage

\setcounter{page}{1} 
\declarationOfOriginality

\cleardoublepage
\tableofcontents
\mainmatter
\chapter{Orgmode Template}
\label{sec:org99c1bb0}
In order to get started with the template you can use the following empty
document and work from there. This is quite a bit redundant, but as a lot of
things are set up with comments and orgmode specific options they won't be
rendered in a Github / Gitlab preview):

\begin{lstlisting}[language=org,numbers=none]
* Document Setup :ignore:
# This section contains all the setup necessary to export this org document
# directly into an ISW style PDF The ~ignore~ tags on the headlines are
# necessary
** Setupfile :ignore:
#+SETUPFILE: settings.org
** Docment Type :ignore:
# % This class does the ISW styling for you (together with scrbook).
# %
# % It handles the following:
# % - Proper input and font encoding (Just type, don't care about the LaTeX
# %   compiler you use or how to type # German umlauts)
# % - Fonts with ligatures and kerning (Tex Gyre fonts are used, part of every
# %   LaTeX installation, text is # nice to read
# % - Bibliography styling for biblatex (declare your bibliography file and you
# %   are ready to go)
# % - Provide command for title page (\makeISWtitle) and declaration of
# %   originality ( \declarationOfOriginality)
# % - Loads packages "biblatex" and "graphics"
#+LATEX_HEADER: \usepackage[
#+LATEX_HEADER:     type=MA
# #+LATEX_HEADER:     type=BA
# #+LATEX_HEADER:     type=FA
# #+LATEX_HEADER:     type=SA
# #+LATEX_HEADER:     type=bachelorproject
#+LATEX_HEADER: ]{iswthesis}
** Language :ignore:
# #+LATEX_HEADER: \usepackage[english, ngerman]{babel}
#+LATEX_HEADER: \usepackage[ngerman, english]{babel}
** Latex: Bibliography :ignore:
#+LATEX_HEADER: \addbibresource{bibliography.bib}
** TODO Document Settings: Author, Title, Date :ignore:
# you need to set both titles, latex and orgmode
#+TITLE: Emacs Orgmode ISW Template
#+LATEX_HEADER:\title{Emacs Orgmode ISW Template}
#+LATEX_HEADER:\titleTranslated{Emacs Orgmode ISW Template}
#+AUTHOR: 
#+LATEX_HEADER:\author{Max Mustermann}
#+LATEX_HEADER:\placeOfBirth{Stuttgart}
#+LATEX_HEADER:\address{Seidenstraße 36, 70174 Stuttgart}
#+LATEX_HEADER:\major{Mechatronik}
#+LATEX_HEADER:\matrnr{1234567}
#+LATEX_HEADER:\date{\today}
#+LATEX_HEADER:\supervisor{My supervisor, M.Sc.}
#+LATEX_HEADER:\professor{Prof. Dr.-Ing. Oliver Riedel}
** Begin Document Section :ignore:

\frontmatter
\makeISWtitle
\cleardoublepage
# % start at page (i) after title page:
\setcounter{page}{1} 
\declarationOfOriginality
# % Kurzfassung/Abstract
\cleardoublepage
\tableofcontents
\mainmatter

* Your Content
* Document Ending: Bibliography, Lists, Appendixes :ignore:
\backmatter
\cleardoublepage
\printbibliography
\cleardoublepage
\listoffigures
\cleardoublepage
\listoftables
\cleardoublepage
\end{lstlisting}
\chapter{(Doom) Emacs specific Configuration}
\label{sec:orgd0a10ad}
In order to get this file working here are the minimally relevant sections from the doom emacs configuration:
\section{init.el}
\label{sec:org2d8626b}

\begin{widemargin}
\begin{lstlisting}[language=Lisp,numbers=none]
(doom! :input

       :completion
       (corfu +orderless)  ; complete with cap(f), cape and a flying feather!
       (helm +childframe +fuzzy +icons)              ; the *other* search engine for love and life

       :ui
       doom              ; what makes DOOM look the way it does
       doom-dashboard    ; a nifty splash screen for Emacs
       (emoji +unicode)  ; 
       hl-todo           ; highlight TODO/FIXME/NOTE/DEPRECATED/HACK/REVIEW
       modeline          ; snazzy, Atom-inspired modeline, plus API
       ophints           ; highlight the region an operation acts on
       (popup +defaults)   ; tame sudden yet inevitable temporary windows
       (vc-gutter +pretty) ; vcs diff in the fringe
       vi-tilde-fringe   ; fringe tildes to mark beyond EOB
       workspaces        ; tab emulation, persistence & separate workspaces

       :editor
       file-templates    ; auto-snippets for empty files
       fold              ; (nigh) universal code folding
       snippets          ; my elves. They type so I don't have to

       :emacs
       dired             ; making dired pretty [functional]
       electric          ; smarter, keyword-based electric-indent
       undo              ; persistent, smarter undo for your inevitable mistakes
       vc                ; version-control and Emacs, sitting in a tree

       :term

       :checkers
       syntax              ; tasing you for every semicolon you forget

       :tools
       (eval +overlay)     ; run code, run (also, repls)
       lookup              ; navigate your code and its documentation
       magit             ; a git porcelain for Emacs
       pdf               ; pdf enhancements

       :os
       (:if (featurep :system 'macos) macos)  ; improve compatibility with macOS

       :lang
       emacs-lisp        ; drown in parentheses
       (latex             ; writing papers in Emacs has never been so fun
        +latexmk
        )
       markdown          ; writing docs for people to ignore
       (org               ; organize your plain life in plain text
        +pretty
        +hugo
        +babel
        +latex
       )
       sh                ; she sells {ba,z,fi}sh shells on the C xor

       :email

       :app

       :config
       (default +bindings +smartparens))
\end{lstlisting}
\end{widemargin}
\section{Defining the scrbook Latex Class}
\label{sec:org3501a55}
In config.el:

\begin{lstlisting}[language=Lisp,numbers=none]
(add-to-list 'org-latex-classes
          '("scrbook"
             "\\documentclass{scrbook}"
             ("\\chapter{%s}" . "\\chapter{%s}")
             ("\\section{%s}" . "\\section*{%s}")
             ("\\subsection{%s}" . "\\subsection*{%s}")
             ("\\paragraph{%s}" . "\\paragraph*{%s}")
             )
     )
\end{lstlisting}
\section{org-special-block-extras}
\label{sec:org8a71643}
Enables interesting stuff such as the \texttt{ignore} tag, which we use extensively in the Document setup section and the \texttt{settings.org} file.

packages.el:
\begin{lstlisting}[language=Lisp,numbers=none]
(package! org-special-block-extras)
\end{lstlisting}

config.el:
\begin{widemargin}
\begin{lstlisting}[language=Lisp,numbers=none]
;; enable ox-extra in order to be able to use the :ignore: tag to ignore the export of headlines
(require 'ox-extra)
(ox-extras-activate '(ignore-headlines))
\end{lstlisting}
\end{widemargin}
\section{Latex Export :: Xelatex}
\label{sec:orga145337}

config.el:
\begin{lstlisting}[language=Lisp,numbers=none]
(after! org
  (setq org-latex-compiler "xelatex")
  (setq org-latex-pdf-process
        '("xelatex -shell-escape -interaction nonstopmode -output-directory %o %f"
          "biber %b"  ;; Run biber to process citations
          "xelatex -shell-escape -interaction nonstopmode -output-directory %o %f"
          "xelatex -shell-escape -interaction nonstopmode -output-directory %o %f"
          "rm -f %o/*.log %o/*.out %o/*.toc %o/*.lof %o/*.lot %o/*.synctex.gz"
          )))
\end{lstlisting}
\section{ox-bibtex}
\label{sec:orgf666890}
config.el:
\begin{lstlisting}[language=Lisp,numbers=none]
(after! org
  (require 'ox-bibtex))
\end{lstlisting}
\section{bibtex citation format}
\label{sec:org8354eb2}
config.el:
\begin{lstlisting}[language=Lisp,numbers=none]
(setq bibtex-completion-format-citation-functions
      '((org-mode      . bibtex-completion-format-citation-org-ref)
        (latex-mode    . bibtex-completion-format-citation-cite)
        (markdown-mode . bibtex-completion-format-citation-pandoc-citeproc)))
\end{lstlisting}
\section{helm-bibtex}
\label{sec:orgfafd4eb}

packages.el:
\begin{lstlisting}[language=Lisp,numbers=none]
(package! helm-bibtex)
\end{lstlisting}

config.el:  
\begin{lstlisting}[language=Lisp,numbers=none]
;; helm-bibtex related stuff
(after! helm
  (use-package! helm-bibtex
    :custom
    ;; The line below tells helm-bibtex to find the path to the pdf
    ;; in the "file" field in the .bib file.
    (bibtex-completion-pdf-field "file"))
  ;; I also like to be able to view my library from anywhere in emacs, for
  ;; example if I want to read a paper. I added the keybind below for that.
  (map! :leader
        :desc "Open literature database"
        "o l" #'helm-bibtex))
\end{lstlisting}
\section{{\bfseries\sffamily TODO} bibliography paths}
\label{sec:org4410698}
You probably want to change this part when starting to use this:

\begin{lstlisting}[language=Lisp,numbers=none]
(setq bibtex-completion-bibliography '("~/uni/thesis/bibliography.bib"))
(setq bibtex-completion-library-path '("~/uni/thesis/literature/"))
;; (setq bibtex-completion-notes-path "~/uni/thesis/literature/")
(setq org-ref-default-bibliography '("~/uni/thesis/bibliography.bib"))
(setq org-ref-pdf-directory "~/uni/thesis/literature/")
(setq org-ref-bibliography-notes "~/uni/thesis/literature.org")
\end{lstlisting}
\section{org-ref}
\label{sec:orgac0e126}

\begin{lstlisting}[language=Lisp,numbers=none]
(package! org-ref)
\end{lstlisting}
\begin{widemargin}
\begin{lstlisting}[language=Lisp,numbers=none]
(defun bibtex-completion-format-citation-org-ref (keys)
  "Format the citation for `org-ref'."
  (mapconcat (lambda (key) (format "[[cite:%s]]" key)) keys ", "))

(use-package! org-ref
    :after org
    :init
    ; code to run before loading org-ref
    :config
    ; code to run after loading org-ref
    (after! org
      (use-package! org-ref
        :config
        ;; Set up citation format for org-mode using org-ref
        (setq bibtex-completion-format-citation-functions
              '((org-mode . bibtex-completion-format-citation-org-ref)
                (latex-mode . bibtex-completion-format-citation-cite)
                (markdown-mode . bibtex-completion-format-citation-pandoc-citeproc))))
      ))
\end{lstlisting}
\end{widemargin}
\section{Code Export}
\label{sec:orgecc9ca5}
In order to export to \texttt{listings} blocks instead of the default \texttt{verbatim} blocks, we need to set this somewhere in config.el:
\begin{lstlisting}[language=Lisp,numbers=none]
(after! org
  (setq org-latex-listings t)
  (setq org-latex-src-block-backend 'listings))
\end{lstlisting}

This way we can then export code blocks such as this:
\begin{lstlisting}[language=org,numbers=none]
#+NAME: lst:bad_code_example
#+CAPTION: "This code does not provide any insights and should not be included."
#+BEGIN_SRC c++ :exports code
#include <iostream>

using namespace std;

int main(void){
    cout << "Hello world." << endl;
    return 0;
}
#+END_SRC
\end{lstlisting}

Into something like this

\begin{lstlisting}[language=c++,label=lst:bad_code_example,caption={``This code does not provide any insights and should not be included.''},captionpos=b,numbers=none]
#include <iostream>

using namespace std;

int main(void){
    cout << "Hello world." << endl;
    return 0;
}
\end{lstlisting}
\chapter{Example Section: Better Margin Control}
\label{sec:orgffd2a0c}
Here are a bunch of examples for the newly added margins. Code blocks show them best, as there is a visible border but they are generic so I've also added a table as well.

In org-mode all of them look more or less like this (\textbf{to prevent the Macro being executed I have added Zero Width Space characters here so copy/pasting this won't work}):
\begin{lstlisting}[language=org,numbers=none]
...
{{{ begin-narrowmargin}}} <- Zero Width Space characters in here

{\{{begin-narrowmargin}}}
#+begin_src emacs-lisp
  (message "Narrow margin example")
#+end_src

| A | B | C | D | E | F | A | B | C | D | E | F | A | B |
|---+---+---+---+---+---+---+---+---+---+---+---+---+---+
| 2 | 8 | 3 | 1 | 3 | 6 | 2 | 8 | 3 | 1 | 3 | 6 | 2 | 8 |
| 5 | 7 | 9 | 2 | 2 | 7 | 5 | 7 | 9 | 2 | 2 | 7 | 5 | 7 |...
{{{ end-narrowmargin}}} <- Zero Width Space characters in here
...
\end{lstlisting}
\section{Narrow Margin (0.5cm)}
\label{sec:org6765ce8}

\begin{narrowmargin}
\begin{lstlisting}[language=Lisp,numbers=none]
  (message "Narrow margin example")
\end{lstlisting}

\begin{center}
\begin{tabular}{rrrrrrrrrrrrrrrrrrrrrrr}
A & B & C & D & E & F & A & B & C & D & E & F & A & B & C & D & E & F & A & B & C & D & E\\
\hline
2 & 8 & 3 & 1 & 3 & 6 & 2 & 8 & 3 & 1 & 3 & 6 & 2 & 8 & 3 & 1 & 3 & 6 & 2 & 8 & 3 & 1 & 3\\
5 & 7 & 9 & 2 & 2 & 7 & 5 & 7 & 9 & 2 & 2 & 7 & 5 & 7 & 9 & 2 & 2 & 7 & 5 & 7 & 9 & 2 & 2\\
\end{tabular}
\end{center}
\end{narrowmargin}
\section{Medium Margin (1cm)}
\label{sec:org5632077}

\begin{mediummargin}
\begin{lstlisting}[language=Lisp,numbers=none]
  (message "Medium margin example")
\end{lstlisting}

\begin{center}
\begin{tabular}{rrrrrrrrrrrrrrrrrrrrrrrr}
A & B & C & D & E & F & A & B & C & D & E & F & A & B & C & D & E & F & A & B & C & D & E & F\\
\hline
2 & 8 & 3 & 1 & 3 & 6 & 2 & 8 & 3 & 1 & 3 & 6 & 2 & 8 & 3 & 1 & 3 & 6 & 2 & 8 & 3 & 1 & 3 & 6\\
5 & 7 & 9 & 2 & 2 & 7 & 5 & 7 & 9 & 2 & 2 & 7 & 5 & 7 & 9 & 2 & 2 & 7 & 5 & 7 & 9 & 2 & 2 & 7\\
\end{tabular}
\end{center}
\end{mediummargin}
\section{Wide Margin (1.5cm)}
\label{sec:orgcc412d2}

\begin{widemargin}
\begin{lstlisting}[language=Lisp,numbers=none]
  (message "Wide margin example")
\end{lstlisting}

\begin{center}
\begin{tabular}{rrrrrrrrrrrrrrrrrrrrrrrrrr}
A & B & C & D & E & F & A & B & C & D & E & F & A & B & C & D & E & F & A & B & C & D & E & F & A & B\\
\hline
2 & 8 & 3 & 1 & 3 & 6 & 2 & 8 & 3 & 1 & 3 & 6 & 2 & 8 & 3 & 1 & 3 & 6 & 2 & 8 & 3 & 1 & 3 & 6 & 2 & 8\\
5 & 7 & 9 & 2 & 2 & 7 & 5 & 7 & 9 & 2 & 2 & 7 & 5 & 7 & 9 & 2 & 2 & 7 & 5 & 7 & 9 & 2 & 2 & 7 & 5 & 7\\
\end{tabular}
\end{center}

\end{widemargin}
\section{Very Wide Margin (2cm)}
\label{sec:orgcb15d45}

\begin{verywidemargin}
\begin{lstlisting}[language=Lisp,numbers=none]
  (message "Very wide margin example")
\end{lstlisting}

\begin{center}
\begin{tabular}{rrrrrrrrrrrrrrrrrrrrrrrrrrr}
A & B & C & D & E & F & A & B & C & D & E & F & A & B & C & D & E & F & A & B & C & D & E & F & A & B & C\\
\hline
2 & 8 & 3 & 1 & 3 & 6 & 2 & 8 & 3 & 1 & 3 & 6 & 2 & 8 & 3 & 1 & 3 & 6 & 2 & 8 & 3 & 1 & 3 & 6 & 2 & 8 & 3\\
5 & 7 & 9 & 2 & 2 & 7 & 5 & 7 & 9 & 2 & 2 & 7 & 5 & 7 & 9 & 2 & 2 & 7 & 5 & 7 & 9 & 2 & 2 & 7 & 5 & 7 & 9\\
\end{tabular}
\end{center}
\end{verywidemargin}
\section{Extra Wide Margin (2.5cm)}
\label{sec:org15d4bdf}

\begin{extrawidemargin}
\begin{lstlisting}[language=Lisp,numbers=none]
  (message "Extra wide margin example")
\end{lstlisting}

\begin{center}
\begin{tabular}{lrrrrrrrrrrrrrrrrrrrrrrrrr}
Gadget & A & B & C & D & E & F & A & B & C & D & E & F & A & B & C & D & E & F & A & B & C & D & E & F & Units\\
\hline
Phone & 2 & 8 & 3 & 1 & 3 & 6 & 2 & 8 & 3 & 1 & 3 & 6 & 2 & 8 & 3 & 1 & 3 & 6 & 2 & 8 & 3 & 1 & 3 & 6 & 1\\
Laptop & 5 & 7 & 9 & 2 & 2 & 7 & 5 & 7 & 9 & 2 & 2 & 7 & 5 & 7 & 9 & 2 & 2 & 7 & 5 & 7 & 9 & 2 & 2 & 7 & 2\\
\end{tabular}
\end{center}
\end{extrawidemargin}
\chapter{Quotes}
\label{sec:org8632686}

Here are some random quotes from the ISW bibliography.

So let's just quote \cite{Kohm2013}

\begin{itemize}
\item \cite{talbot2014}

\item Talbot, N. (2014). User manual for glossaries.sty v4.09.

\item A quote with page number: \cite{talbot2014}{[}p.10] (not sure if this one works as in intended)

\begin{itemize}
\item \cite{Hoffmann2014}
\end{itemize}
\end{itemize}
\chapter{Import of the rest of the chapters}
\label{sec:org33e130c}
Lets add in the other chapters here:

\begin{lstlisting}[language=org,numbers=none]
#+LATEX: \input{chapters/Introduction}
#+LATEX: \input{chapters/Examples}
#+LATEX: \input{chapters/Tooling}

#+LATEX: \input{chapters/Acronyms}
#+LATEX: \input{chapters/Symbols}
\end{lstlisting}

When Importing everything this way, we get a bunch of errors / warnings. As they
are caused by the template itself I won't bother to fix them here. Commenting
the \texttt{\#+LATEX} lines out will compile a working document.

\input{chapters/Introduction}
\input{chapters/Examples}
\input{chapters/Tooling}

\input{chapters/Acronyms}
\input{chapters/Symbols}
\backmatter
\cleardoublepage
\printbibliography
\cleardoublepage
\listoffigures
\cleardoublepage
\listoftables
\cleardoublepage
\end{document}
